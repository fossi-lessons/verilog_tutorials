\documentclass{article}
\usepackage{hyperref}
\usepackage[utf8]{inputenc}
\usepackage{setspace}
\usepackage[lmargin=1in,rmargin=1in,tmargin=1in,bmargin=1in]{geometry}
\setlength{\marginparwidth}{.9in}
\linespread{1.25}

\begin{document}
\title{Exercise 2: State Machines}
\date{}
\maketitle

\section*{Overview}
In this exercise, we will be building a simple counter for the number of cars in
a parking lot based on the status of two signals. It will involve working with
behavioral Verilog as well as implementing a state machine, which a very common
design pattern used in hardware.

\subsection*{Mechanical Goals}
By the end of going through this exercise, you should:
\begin{itemize}
    \item Be comfortable implementing both sequential and combinational logic in
    synthesizable behavioral Verilog
    \item Know how to implement a state machine with good style practices
\end{itemize}

\subsection*{Conceptual Goals}
\begin{itemize}
    \item Understand the difference between combinational and sequential logic
    \item Understand the concept of a state machine and how to design a basic
    state machine 
\end{itemize}

\section*{Problem}
Implement the hardware to maintain a count of cars in a parking lot based on the
inputs from two sensors. There are two sensors, an outer sensor and an inner
sensor. A car has entered after the pattern \{\texttt{outer, inner}\} = \{1,0; 1,1;
0,1; 0,0\} has been seen. A car exiting is the pattern \{\texttt{outer, inner}\} =
\{0,1; 1,1; 1,0; 0,0\}. Cars may stop, but they will not change direction, so they
may spend multiple cycles at any part of the pattern, but they will not go
backwards in the pattern.  The counter should not be changed until the full sensor
pattern has been seen.

\subsection*{Files}
\begin{itemize}
    \item \texttt{lot\_counter\_pkg.sv}: Used for definitions of constants
    \item \texttt{lot\_counter\_state\_logic.sv}: The bulk of the logic should
    be implemented in this file. This will contain all the combinational logic
    for the state machine
    \item \texttt{lot\_counter\_state\_regs.sv}: This is the file that contains
    all the sequential logic needed. Do not change this file. It has a register
    to store the current count of cars and a register to store the current
    state.
    \item \texttt{lot\_counter\_top.sv}: This file should instantiate both the
    state logic and the state regs and is the top-level wrapper for simulation
\end{itemize}

\section*{Steps}
\textbf{Step 0: Design your state machine}\\
Outline the states that you want and what each of them should do and then fill
in the enum in \texttt{log\_counter\_pkg.sv}. The first state (\texttt{READY})
that the state machine will initialize to has already been provided. 

Think about the states that you go through when you see the entry and exit
patterns and what signals cause you to transition states. Also consider when to
set signals going from the combinational logic file to the sequential logic
file. There are formal diagrams to do this (see ASM charts), but whatever type
of diagram or outline is going to work for you is good.

\noindent\textbf{Step 1: Implement the combinational logic}\\
In \texttt{lot\_counter\_state\_logic.sv}, implement the logic for the state
machine following the design pattern shown in the state machine section. Use a
case statement and make sure that every signal used in the \texttt{always} block
is assigned a value on every path through the code. 

\noindent\textbf{Step 2: Instantiate the state logic module}\\
In \texttt{lot\_counter\_top.sv}, instantiate the
\texttt{log\_counter\_state\_logic} module and connect the ports to the
appropriate signals.

\noindent\textbf{Step 3: Build and test}\\
As before, run \texttt{make build} to compile the design. Then \texttt{make run}
to actually execute the tests. Running \texttt{make all} will do both steps.

Running the tests will produce a waveform for debugging in
\texttt{logs\\vlt\_dump.vcd} as well as print out some state of the design. Feel
free to edit \texttt{sim\_main.cpp} to add more tests or print out different
state of the design.
    
\end{document}