
\documentclass{article}
\usepackage{hyperref}
\usepackage[utf8]{inputenc}
\usepackage{setspace}
\usepackage[lmargin=1in,rmargin=1in,tmargin=1in,bmargin=1in]{geometry}
\setlength{\marginparwidth}{.9in}
\linespread{1.25}

\begin{document}
\title{Exercise 3: Interfaces \& Latency}
\date{}
\maketitle

\section*{Overview}
In this exercise, we will be building two designs. The first will be a
multi-cycle implementation of a multiplier. The second will be building a RAM
that can support reading and writing the same address in the same cycle. We'll
be working with latency insensitive interfaces, which is a commonly used design
pattern in more complex hardware designs.

\section*{Background}
\subsection*{Interface Latency}
When you build an interface for a hardware module, you not only have an
expectation of what data is provided by each port, but also an expectation
of when the values on the port are valid. 

Consider the multiplier we'll be implementing and let's imagine it's a 32-bit
multiplier. Doing the multiplication in a single cycle is impractical from a
hardware perspective (consider what the combinational circuit for a 32-bit
multiplication might look like). So we want to be able to take multiple cycles
to do the multiplication. So when does another piece of logic using the module
know when the output is valid?

One way could be to always wait a certain number of cycles. How many cycles though?
Well, if we process the multiplier value one bit at a time, adding the
multiplicand to a total depending on whether the bit is 1 or a 0, it could take 32
cycles. But do we always need to take 32 cycles? Consider if we're multiplying
by 4 versus $2^31$. And what if we want to switch to a different implementation
where we process the multiplier value multiple bits at the time? That could also
change the timing of the output, so that's not very helpful.

While we're at it, let's consider how the multiplier knows to start a
multiplication. Should the multiplier just always use the values on its input
after it finishes the multiplication? What if the values on its input change
during the multiplication? Should it start over? Or should it finish the
multiplication its doing. And furthermore, now the module responsible for
providing the values as input to the multiplier might need to track how many
cycles the multiplier takes to finish. 

This is only one module too. Imagine where we have to construct a whole
system where modules may have different interface latencies. This would get
pretty out of hand pretty quickly.

\subsection*{Latency-Insensitive Interfaces}
Enter: latency insensitive interfaces. In this design pattern, modules provide
an interface that provides a simple standardized handshake. Values are exchanged
at the interface only in the cycle the handshake occurs. This allows us easily
understand when a consumer is able to accept new values and when a producer is
providing valid values for computation. 

The typical form of this interface uses a val-rdy (valid-ready) handshake. This
means that modules have (at least) the valid and ready control signals
associated with their interface. The producer of some values will set the valid
signal and the consumer will set the ready signal. If both signals are high in
the same cycle, then the exchange of values is complete, and the consumer has
taken the values. Now we no longer have to remember how many cycles a module may
take to produce a value, for example, because it will just set the valid signal
when it is finished.

Let's consider the handshake in the context of providing values to the
multiplier. The logic providing the values will set its valid signal high when
it has values that it wants to request the multiplier will operate on. In this
case, the logic is the producer and the multiplier is the consumer. If the
multiplier is ready to operate on a new set of values, it will its ready signal
high. If both the valid and ready signals are high in the same cycle, then the
multiplier will start to operate on those values.

Since the multiplier also produces an output, it will have a val-rdy interface
in which it is the producer and some other logic is the consumer. So when it has
calculated the product, it can set its output valid signal high. Once the logic
consuming the value sets its ready signal high, the multiplier knows the product
has been consumed, and it can start another transaction.

\subsection*{Synchronous vs Asynchronous Memories}
While we're talking about latency and interfaces, let's touch on memories,
specifically SRAM or register files or or other similar things. Not DRAM. DRAM
is a whole other thing.  Memory primitives typically don't provide a latency
insensitive interface, but they come in two main variants: asynchronous and
synchronous. These variants refer to the latency of a read on the memory. That
is, when you make a read request, when is the value on the output the value you
requested.

For \textbf{asynchronous memories}, the value is available in the same cycle you
requested in. This is the behavior of a register file. For \textbf{synchronous
memories}, the value is available in the next cycle after you requested it. You
can imagine a register file that has an extra flop on either the input or the
output.

\section*{Part 1: the multiplier}
The task is to implement a multi-cycle multiplier with a parameterizable operand
width. It should have a val-rdy interface. 
\subsection*{Files}
\begin{itemize}
    \item \texttt{multiplier\_top.sv}: instantiates the multiplier and is the
    top-level for simulation
    \item \texttt{multiplier.sv}: multiplier top-level. Look here to see the
    val-rdy interface
    \item \texttt{multiplier\_ctrl.sv}: empty file to implement the multiplier
    control logic
    \item \texttt{multiplier\_data.sv}: empty file to implement the multiplier
    datapath logic
\end{itemize}

\subsection*{Implementation}
As hinted at above, there are multiple ways to do this, including optimizations.
We don't need to get fancy with the actual multiplier logic here. The point is
to provide a simple example of a latency-sensitive interface. However, your
implementation shouldn't always take the same number of cycles to output the
product.

One way to do this could be to start with iterating over the multiplier one bit
at a time and adding the multiplicand to a running sum and always iterating over
the whole multiplier. The next step could be to terminate when you don't need to
iterate over the multiplier any more and provide the product as soon as it's
ready. 

\section*{Part 2: the memory}
The task here is to take the provided synchronous 1r1w memory primitive and
build another synchronous 1r1w memory but with some extra features. A 1r1w
memory means that there is one port for issuing write requests and another for
issuing read requests, so it can accept both a read and write request in the
same cycle. The exception is when there is a read and write request for the same
address in the same cycle. In this case, the primitive throws an error.

There are two features to add. The first is that we want to provide a val-rdy
interface. The second is that we want to be able to handle reads and writes to
the same address in the same cycle by bypassing the written value.

\subsection*{Files}
\begin{itemize}
    \item \texttt{mem\_wr\_bypass\_top.sv}: top-level for simulation,
    instantiates the memory
    \item \texttt{mem\_1r1w\_sync\_wr\_bypass.sv}: memory top-level. It should
    contain your logic to implement the val-rdy interface as well as the
    instantiation of the provided memory primitive
    \item \texttt{mem\_1r1w\_sync.sv}: provided memory primitive. Do not edit
    this file.
\end{itemize}

\subsection*{Implementation}
Like the multiplier, the logic here shouldn't be very involved and is more
concerned with your understanding of how to implement and use a val-rdy
interface.

Start with implementing the val-rdy interface. There is both a val-rdy handshake
on the request side and response side. In particular, this means that the
response may not be consumed the cycle after, and your logic must make sure
somehow that the response remains valid, even if there is another valid request
offered on the request interface.

After you've gotten that working, then move onto implementing the bypassing.
Consider what you need to save in order to bypass the correct data from write to
read as well as how to tell whether you should use the data you've saved or the
data from the memory.

\end{document}